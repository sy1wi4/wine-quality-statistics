\documentclass{article}

\usepackage{polski}
% skomentowana dla MAC wersja {inputenc}
 \usepackage[utf8]{inputenc}
% \usepackage[T1]{fontenc}
%\usepackage[cp1250]{inputenc}
\usepackage{polski}
\usepackage{amssymb}
\usepackage{color}
\usepackage{amsmath}
\usepackage{Sweave}
\usepackage{enumerate}
\usepackage{hyperref}
\usepackage[version=4]{mhchem}
\usepackage{geometry}

\newgeometry{tmargin=2cm, bmargin=2cm, lmargin=1cm, rmargin=1cm}

\title{Analiza jakości wina czerwonego i jego właściwości fizykochemicznych}
\author{\textbf{402 332, Sylwia MAREK}, czwartek $14^{40}$\\ 
\textit{AGH, Wydział Informatyki Elektroniki i Telekomunikacji}\\
\textit{Rachunek prawdopodobieństwa i statystyka 2020/2021}}
\date{Kraków, \today}


\begin{document}

% potrzebne biblioteki 





\input{winoRaportPls-concordance}
\maketitle

\textit{Ja, niżej podpisany(na) własnoręcznym podpisem deklaruję, że przygotowałem(łam) przedstawiony do oceny projekt samodzielnie i żadna jego część nie jest kopią pracy innej osoby.}
\begin{flushright}
{............................................}
\end{flushright}

\section{Streszczenie raportu}
Raport powstał w oparciu o analizę danych dotyczących portugalskiego czerwonego wina "Vinho Verde".
Dotycza one jego jakosci oraz wlasciwosci fizykochemicznych opisanych pokrotce w dalszej czesci raportu.

W trakcie analizy zbadano:
\begin{itemize}
\item Podstawowe wlasnosci danych
\item Co wplywa w glownej mierze na jakosc wina?
\item Rodzaj rozkladu zmiennych 
\item Czy ktoras ze zmiennych posiada powszechny rozklad?
\item Zależnosci miedzy danymi ze szczegolnym podkresleniem zaleznosci  liniowych
\item Miedzy ktorymi zmiennymi wystepuje korelacja?
\end{itemize}

\section{Opis danych}

Dane do projektu pochodzą ze strony \href{url}{\texttt{https://www.kaggle.com/}}. 



Badane dane zawieraja nastepujace zmienne (numeryczne):
\begin{enumerate}
\item fixed.acidity
\item volatile.acidity - zawartosc kwasu octowego, wysoka odpowiada za nieprzyjemny, octowy posmak
\item citric.acid - zawartosc kwasu cytrynowego, w malych ilosciach dodaje "swiezosci"
\item residual.sugar - zawartosc cukru pozostalego po zakonczeniu fermentacji,$ > 45g/l \Rightarrow$ wino uznawane za slodkie
\item chlorides - zawartosc soli
\item free.sulfur.dioxide - wolny dwutlenek siarki (\ce{SO2}), zapobiega rozwojowi drobnoustrojow i utlenianiu wina
\item total.sulfur.dioxide - wolny i zwiazany \ce{SO2} lacznie, w malych ilosciach niewykrywalny
\item density - gestosc, zblizona do gestosci wody, zalezna od zawartosci alkoholu oraz cukru
\item pH - skala zasadowosci/kwasowosci - od 0 (bardzo kwasowy) do 14 (bardzo zasadowy); wino zazwyczaj 3-4 pH
\item sulphates - siarczany - dodatek, mogacy przyczyniac sie do wzrostu zawartosci \ce{SO2}, dziala przeciwbakteryjnie
\item alcohol - zawartosc procentowa alkoholu
\item quality - jakosc wina, w skali $[0-10]$)
\end{enumerate}

Moznaby spodziewac sie m.in., ze im wieksza zawartosc kwasu octowego (volatile.acidity, tym jakosc wina nizsza)


\section{Analiza danych}

Poniżej zamieszczono przykładowe wywołania prostych formuł z pakietu R, których składnia może przydać się w projekcie.
\begin{enumerate}
\item wczytanie danych
\begin{Schunk}
\begin{Sinput}
> wine <- read.csv("D:/studia/RPIS/projekt/winequality.csv", sep=";")
\end{Sinput}
\end{Schunk}

\item funkcja wyliczajaca mode
\begin{Schunk}
\begin{Sinput}
> Mode <- function(v) {
+   uniqv <- unique(v)
+   uniqv[which.max(tabulate(match(v, uniqv)))]
+ }
\end{Sinput}
\end{Schunk}

\item funkcja obliczajaca, ile jest wartosci odstajacych
\begin{Schunk}
\begin{Sinput}
> outliers <- function(x){
+   length(which(x >  mean(x) + 3 * sd(x) | x < mean(x) - 3 * sd(x))  ) / length(x)
+ }
\end{Sinput}
\end{Schunk}




\end{enumerate}

\subsection{Weryfikacja poprawnosci danych}
\begin{Schunk}
\begin{Sinput}
> anyNA(wine)
\end{Sinput}
\begin{Soutput}
[1] FALSE
\end{Soutput}
\end{Schunk}

W analizowanym zbiorze nie ma brakujacych danych.
\newline
Wszelkie wartosci odstajace beda brane pod uwage, zadnych nie uswaamy.

\subsection{Wydobywanie podstawowych informacji z danych}

\begin{Schunk}
\begin{Sinput}
> summary(wine)
\end{Sinput}
\begin{Soutput}
 fixed.acidity   volatile.acidity  citric.acid    residual.sugar  
 Min.   : 4.60   Min.   :0.1200   Min.   :0.000   Min.   : 0.900  
 1st Qu.: 7.10   1st Qu.:0.3900   1st Qu.:0.090   1st Qu.: 1.900  
 Median : 7.90   Median :0.5200   Median :0.260   Median : 2.200  
 Mean   : 8.32   Mean   :0.5278   Mean   :0.271   Mean   : 2.539  
 3rd Qu.: 9.20   3rd Qu.:0.6400   3rd Qu.:0.420   3rd Qu.: 2.600  
 Max.   :15.90   Max.   :1.5800   Max.   :1.000   Max.   :15.500  
   chlorides       free.sulfur.dioxide total.sulfur.dioxide    density      
 Min.   :0.01200   Min.   : 1.00       Min.   :  6.00       Min.   :0.9901  
 1st Qu.:0.07000   1st Qu.: 7.00       1st Qu.: 22.00       1st Qu.:0.9956  
 Median :0.07900   Median :14.00       Median : 38.00       Median :0.9968  
 Mean   :0.08747   Mean   :15.87       Mean   : 46.47       Mean   :0.9967  
 3rd Qu.:0.09000   3rd Qu.:21.00       3rd Qu.: 62.00       3rd Qu.:0.9978  
 Max.   :0.61100   Max.   :72.00       Max.   :289.00       Max.   :1.0037  
       pH          sulphates         alcohol         quality     
 Min.   :2.740   Min.   :0.3300   Min.   : 8.40   Min.   :3.000  
 1st Qu.:3.210   1st Qu.:0.5500   1st Qu.: 9.50   1st Qu.:5.000  
 Median :3.310   Median :0.6200   Median :10.20   Median :6.000  
 Mean   :3.311   Mean   :0.6581   Mean   :10.42   Mean   :5.636  
 3rd Qu.:3.400   3rd Qu.:0.7300   3rd Qu.:11.10   3rd Qu.:6.000  
 Max.   :4.010   Max.   :2.0000   Max.   :14.90   Max.   :8.000  
\end{Soutput}
\end{Schunk}

\begin{Schunk}
\begin{Sinput}
> head(wine)
\end{Sinput}
\begin{Soutput}
  fixed.acidity volatile.acidity citric.acid residual.sugar chlorides
1           7.4             0.70        0.00            1.9     0.076
2           7.8             0.88        0.00            2.6     0.098
3           7.8             0.76        0.04            2.3     0.092
4          11.2             0.28        0.56            1.9     0.075
5           7.4             0.70        0.00            1.9     0.076
6           7.4             0.66        0.00            1.8     0.075
  free.sulfur.dioxide total.sulfur.dioxide density   pH sulphates alcohol
1                  11                   34  0.9978 3.51      0.56     9.4
2                  25                   67  0.9968 3.20      0.68     9.8
3                  15                   54  0.9970 3.26      0.65     9.8
4                  17                   60  0.9980 3.16      0.58     9.8
5                  11                   34  0.9978 3.51      0.56     9.4
6                  13                   40  0.9978 3.51      0.56     9.4
  quality
1       5
2       5
3       5
4       6
5       5
6       5
\end{Soutput}
\end{Schunk}


Przyjrzyjmy sie wykresowi slupkowemu, przedstawiajacemu  licznosc poszczegolnych grup wina, 
ze wzgledu na jego jakosc - jest ona okreslona jako liczba calkowita
z przedzialu $[0,10]$, im wysza ocena, tym lepsze jakosciowo jest wino

\begin{Schunk}
\begin{Sinput}
> data <- as.data.frame(table(wine$quality))
> data
\end{Sinput}
\begin{Soutput}
  Var1 Freq
1    3   10
2    4   53
3    5  681
4    6  638
5    7  199
6    8   18
\end{Soutput}
\begin{Sinput}
> ggplot(data, aes(x=Var1, y =Freq)) + geom_bar(stat = "identity", fill = c("#66CC99")) +
+   geom_text(aes(label = Freq), nudge_y = 20) + xlab("quality") + ylab("count")
\end{Sinput}
\end{Schunk}
\includegraphics{winoRaportPls-quality_barplot}


Zauwazmy, ze najbardziej liczna jest grupa win o jakosci 5 (w skali 0-10), najnizsza wystepujaca jakosc wina to 3, 
najwyzsza zas - 8 (18 win najwyzszej jakosci)
\newline
Wina "średniej" jakosci (5/6) wystepuja najczesciej, duzo rzadziej wina wysokiej i niskiej 
jakosci (najrzadziej niskiej)



Przyjrzyjmy sie wykresom pudelkowym i/lub histogramom zmiennych:

\subsubsection{QUALITY}
\begin{Schunk}
\begin{Sinput}
> ggplot(data = wine, mapping = aes(y = quality)) + 
+     geom_boxplot(width = 1.2, fill = c("#CC99FF"), outlier.size = 2, 
+                  alpha = 0.5, size = 0.6) +
+     ylab("Quality") + coord_flip() + stat_boxplot(geom = 'errorbar')
\end{Sinput}
\end{Schunk}
\includegraphics{winoRaportPls-quality_boxplot}



\subsubsection{PH}
\begin{Schunk}
\begin{Sinput}
> ggplot(data = wine, mapping = aes(y = pH)) + 
+   geom_boxplot(width = 1.2, fill = c("#CC99FF"), outlier.size = 2, 
+                alpha = 0.5, size = 0.6) +
+   ylab("pH") + coord_flip() + stat_boxplot(geom = 'errorbar')
\end{Sinput}
\end{Schunk}
\includegraphics{winoRaportPls-ph_boxplot}

Na podstawie wykresu mozna podejrzewac, ze rozklad ten jest symetryczny, moze normalny?
\newline
Spojrzmy zatem na histogram z naniesiona gestoscia tegoz rozkladu oraz gestoscia rozkladu normalnego

\begin{Schunk}
\begin{Sinput}
> ggplot(data = wine, aes(pH)) + 
+   geom_histogram(aes(y = ..density..), binwidth = 1/50, color = "darkblue", fill = c("#99CCFF")) +
+   geom_density(size = 0.9, color = c("#000033")) + 
+   stat_function(fun = dnorm, args = list(mean = mean(wine$pH), sd = sd(wine$pH)), color = "red", size = 0.7)
\end{Sinput}
\end{Schunk}
\includegraphics{winoRaportPls-ph_hist}



\subsubsection{ALCOHOL}

\begin{Schunk}
\begin{Sinput}
> ggplot(data = wine, mapping = aes(y = alcohol)) + 
+   geom_boxplot(width = 1.2, fill = c("#CC99FF"), outlier.size = 2, 
+                alpha = 0.5, size = 0.6) +
+   ylab("Alcohol") + coord_flip() + stat_boxplot(geom = 'errorbar')
\end{Sinput}
\end{Schunk}
\includegraphics{winoRaportPls-alcohol_boxplot}

Podejrzewamy skosnosc prawostronna rozkladu.
Sprawdzmy:

\begin{Schunk}
\begin{Sinput}
> skewness(wine$alcohol)
\end{Sinput}
\begin{Soutput}
[1] 0.8600211
\end{Soutput}
\end{Schunk}

Wspolczynnik skosnosci $>0$, co swiadczy o prawostronnej asymetrii rozkladu (rozklad dodatnio skosny).

Spojrzmy jeszcze jak przedstawia sie histogram.

\begin{Schunk}
\begin{Sinput}
> ggplot(data = wine, aes(alcohol)) + 
+   geom_histogram( color = "darkblue", fill = c("#99CCFF"), binwidth = 1/5)
\end{Sinput}
\end{Schunk}
\includegraphics{winoRaportPls-alcohol_hist}

Najczesciej wystepujaca zawartosc alkoholu oscyluje miedzy 9.5 a 11.1 $\%$
im wieksza zawartosc wina, tym rzedziej ono wystepuje


\subsubsection{DENSITY}

\begin{Schunk}
\begin{Sinput}
> ggplot(data = wine, mapping = aes(y = density)) + 
+   geom_boxplot(width = 1.2, fill = c("#CC99FF"), outlier.size = 2, 
+                alpha = 0.5, size = 0.6) +
+   ylab("Density") + coord_flip() + stat_boxplot(geom = 'errorbar')
\end{Sinput}
\end{Schunk}
\includegraphics{winoRaportPls-density_boxplot}

Rozklad symetryczny, normalny?

Popatrzmy znow na histogram z naniesiona gestoscia tegoz rozkladu oraz gestoscia rozkladu normalnego :

\begin{Schunk}
\begin{Sinput}
> ggplot(data = wine, aes(density)) + 
+   geom_histogram(aes(y = ..density..), binwidth = 1/4000, color = "darkblue", fill = c("#99CCFF")) +
+   geom_density(size = 0.9, color = c("#000033")) + 
+   stat_function(fun = dnorm, args = list(mean = mean(wine$density), sd = sd(wine$density)), color = "red", size = 0.7)
\end{Sinput}
\end{Schunk}
\includegraphics{winoRaportPls-density_boxplot}

\subsubsection{SULPHATES}

\begin{Schunk}
\begin{Sinput}
> ggplot(data = wine, mapping = aes(y = sulphates)) + 
+   geom_boxplot(width = 1.2, fill = c("#CC99FF"), outlier.size = 2, 
+                alpha = 0.5, size = 0.6) +
+   ylab("Sulphates") + coord_flip() + stat_boxplot(geom = 'errorbar')
\end{Sinput}
\end{Schunk}
\includegraphics{winoRaportPls-sulphates_boxplot}


\begin{Schunk}
\begin{Sinput}
> ggplot(data = wine, aes(sulphates)) + 
+   geom_histogram(aes(y = ..density..), binwidth = 1/50, color = "darkblue", fill = c("#99CCFF")) +
+   geom_density(size = 0.9, color = c("#000033")) 
\end{Sinput}
\end{Schunk}
\includegraphics{winoRaportPls-suphates_hist}

Rozklad ten jest wyraznie prawostronnie skosny.


\subsubsection{RESIDUAL SUGAR}

\begin{Schunk}
\begin{Sinput}
> ggplot(data = wine, aes(residual.sugar)) + 
+   geom_histogram( color = "darkblue", fill = c("#99CCFF"), binwidth = 1/4) 
\end{Sinput}
\end{Schunk}
\includegraphics{winoRaportPls-sugar_hist}


\subsubsection{CITRIC ACID}

\begin{Schunk}
\begin{Sinput}
> ggplot(data = wine, aes(citric.acid)) + 
+   geom_histogram( color = "darkblue", fill = c("#99CCFF"), binwidth = 1/100)
\end{Sinput}
\end{Schunk}
\includegraphics{winoRaportPls-citr_hist}

Na wykresie widac 3 interesujace maksima lokalne w okolicy 0, 0.25 oraz 0.5 $g$. 
Co ma na to wplyw?



\subsubsection{Obserwacje}

Wiekoszosc badanych cech ma wiele wartosci odstajacych
zauwazyc mozna, ze pH oraz gestosc maja rozklad symetryczny
by sprawdzic, czy jest to rozklad normalny (a nie plato- lub leptokurtyczny)
mozemy posluzyc sie obliczeniem kolejno sredniej arytm., mediany oraz mody  
\newline
Wiemy, ze rozklad dla rozkladu normalnego mamy jedno maksimum, kurtoza wynosi 0,
w punkci centralnym znajduja sie srednia arytm., mediana, moda (dominanta)
zatem sprawdzmy je:


{\bf{pH}}
\begin{Schunk}
\begin{Sinput}
> ph_mean = mean(wine$pH)
> ph_med = median(wine$pH)
> ph_mode = Mode(wine$pH)
> cat("mean   ", ph_mean, "\nmedian:", ph_med, "\nmode   ", ph_mode)
\end{Sinput}
\begin{Soutput}
mean    3.311113 
median: 3.31 
mode    3.3
\end{Soutput}
\end{Schunk}

Kurtoza (miara splaszczenia wartosci rozkladu cechy):

\begin{Schunk}
\begin{Sinput}
> kurtosis(wine$pH)
\end{Sinput}
\begin{Soutput}
[1] 0.8006714
\end{Soutput}
\end{Schunk}

kurtoza $> 0 \Rightarrow$ rozklad wysmukly (wartosci bardziej skoncentrowane niz przy 
rozkladzie normalnym)

Dla pewnosci uzyjmy testu $Shapiro-Wilka$
\newline
Testować będziemy hipotezę zerową {\bf{H0}}: pH pochodzi z rozkładu zbliżonego do normalnego, wobec hipotezy alternatywnej {\bf{H1}}: pH nie pochodzi z rozkładu zbliżonego do normalnego
Jeżeli $p-wartosc > 0.05$, to przyjmujemy {\bf{H0}} 

\begin{Schunk}
\begin{Sinput}
> shapiro.test(wine$pH)
\end{Sinput}
\begin{Soutput}
	Shapiro-Wilk normality test

data:  wine$pH
W = 0.99349, p-value = 1.712e-06
\end{Soutput}
\end{Schunk}

A wiec odrzucamy hipoteze zerowa, a przyjmujemy alternatywna, co za tym idzie - rozklad ten nie jest normalny.
Jednak zauwazamy podobienstwo do tegoz rozkladu.


{\bf{density}}

\begin{Schunk}
\begin{Sinput}
> den_mean = mean(wine$density)
> den_med = median(wine$density)
> den_mode = Mode(wine$density)
> cat("mean   ", den_mean, "\nmedian:", den_med, "\nmode   ", den_mode)
\end{Sinput}
\begin{Soutput}
mean    0.9967467 
median: 0.99675 
mode    0.9972
\end{Soutput}
\end{Schunk}

Kurtoza:
\begin{Schunk}
\begin{Sinput}
> kurtosis(wine$density)
\end{Sinput}
\begin{Soutput}
[1] 0.9274108
\end{Soutput}
\end{Schunk}

kurtoza $> 0 \Rightarrow$ rozklad wysmukly (wartosci bardziej skoncentrowane niz przy 
rozkladzie normalnym)

Testu $Shapiro-Wilka$

\begin{Schunk}
\begin{Sinput}
> shapiro.test(wine$density)
\end{Sinput}
\begin{Soutput}
	Shapiro-Wilk normality test

data:  wine$density
W = 0.99087, p-value = 1.936e-08
\end{Soutput}
\end{Schunk}

Znowu nie jest to rozklad normalny, mimo ze z wykresu widac pewne podobienstwo.



Przyjrzyjmy sie jeszcze wykresom kwantyl-kwantyl (Q-Q Plot) tych dwoch zmiennych:
(Q-Q Plot - sluzy do porownywania rozkladu w probie z wybranym rozkladem hipotetycznym, w tym wypadku rozkladem normalnym)


{\bf{pH}}

\begin{Schunk}
\begin{Sinput}
> ggplot(wine, aes(sample = pH)) + 
+   stat_qq(color = c("#FF6633"), size = 1.5) +
+   stat_qq_line(color = c("#6633CC"), size = 0.8) +
+   ggtitle("Q-Q Plot") +
+   theme(plot.title = element_text(hjust = 0.5))
\end{Sinput}
\end{Schunk}
\includegraphics{winoRaportPls-ph_qqplot}

{\bf{density}}
\begin{Schunk}
\begin{Sinput}
> ggplot(wine, aes(sample = density)) + 
+   stat_qq(color = c("#FF6633"), size = 1.5) +
+   stat_qq_line(color = c("#6633CC"), size = 0.8) +
+   ggtitle("Q-Q Plot") +
+   theme(plot.title = element_text(hjust = 0.5))
\end{Sinput}
\end{Schunk}
\includegraphics{winoRaportPls-denisty_qqplot}


Zauwazyc mozna ze rozklad pH jest bardzo zblizony do normalnego, jednak 
rozni sie przez wartosci odstajace, ktorych polozenie znacznie odbiega od prostej
rozklad density odbiega jeszcze bardziej
\newline
Sprawdzmy, jak wiele jest wartosci odstajacych dla zmiennych pH oraz gestosc:


\begin{Schunk}
\begin{Sinput}
> outliers(wine$density)
\end{Sinput}
\begin{Soutput}
[1] 0.01125704
\end{Soutput}
\begin{Sinput}
> outliers(wine$pH)
\end{Sinput}
\begin{Soutput}
[1] 0.005003127
\end{Soutput}
\end{Schunk}




\subsection{Korelacja $r-Pearsona$}
Korelacja sluzy do sprawdzenia, czy zmienne ilosciowe sa
powiazane zwiazkiem liniowym
\newline
Correlogram - graficznie przedstawiona macierz korelacji

Na ponizszym wykresie przedstawione jest polaczenie correlogramu z testem istotnosci - za wynik 
nieistotny statystycznie uznajemy taki, ze p-wartosc > 0.05
(oznaczone jako puste pole na wykresie)


\begin{Schunk}
\begin{Sinput}
> corr <- cor(wine)
> pval <- cor_pmat(wine)
> ggcorrplot(corr, hc.order = TRUE, type = "upper", lab = TRUE,
+            colors = c("#2166AC", "white", "#CC0000"), p.mat = pval, 
+            sig.level = 0.05, insig = "blank", ggtheme = ggplot2::theme_gray)
\end{Sinput}
\end{Schunk}
\includegraphics{winoRaportPls-correlogram}


Skupmy sie na tym, od czego w glownej mierze zalezy jakosc wina
jak widzimy na wykresie, najwieksza korelacja (liniowa) wystepuje kolejno 
dla: alcohol, volatile.acidity, sulphates, citric.acid, etc.

Spojrzmy jak zmienia sie zawartosc alkoholu wraz ze wzrostem jakosci :

\begin{Schunk}
\begin{Sinput}
> ggplot(data = wine, mapping = aes(x = as.factor(quality), y = alcohol)) + 
+   geom_boxplot(fill = c("#CC99FF"), outlier.size = 2, alpha = 0.5, size = 0.6) +
+   ylab("Alcohol") + xlab("Quality") + stat_boxplot(geom = 'errorbar') 
\end{Sinput}
\end{Schunk}

Zauwazyc mozna, ze im wyzsza jakosc wina, tym wieksza mediana zawartosci alkoholu,
wyjatkiem jest wino jakosci 5, tam tez pojawia sie spora ilosc wartosci odstajacych



Przeanalizujmy jeszcze zawartosc kwasu octowego (volatile.acidity), by sprawdzic,
czy poczatkowe przypuszczenie, jakoby wieksza jego zawartosc powodowala spadek 
jakosci, jest sluszne

\begin{Schunk}
\begin{Sinput}
> ggplot(data = wine, mapping = aes(x = as.factor(quality), y = volatile.acidity)) + 
+   geom_boxplot(fill = c("#CC99FF"), outlier.size = 2, alpha = 0.5, size = 0.6) +
+   ylab("Volatile acidity") + xlab("Quality") + stat_boxplot(geom = 'errorbar') 
\end{Sinput}
\end{Schunk}
\includegraphics{winoRaportPls-col_quality_boxplots}

Widac wyraznie, ze im wieksza zawartosc kwasu octowego w winie, tym gorsza jest jego jakosc - co zgadza sie z przypuszczeniami

\vspace{1cm}
Stosunkowo duza korelacja (ujemna) wystepuje pomiedzy zawartoscia kwasu octowego a kwasu cytrynowego (-0.55), a takze miedzy kazdym z nich oraz jakoscia wina

Sprawdzmy jak mediana (ktora odporna jest na wartosci odstajace) zawartosci 
kazdego z powyzszych kwasow zmienia sie wraz ze wzrostem jakosci wina


\begin{Schunk}
\begin{Sinput}
> ggplot(wine, aes(quality, citric.acid)) +
+   stat_summary(fun.y = median, color = "lightsalmon", geom = "line", size =1) 
\end{Sinput}
\end{Schunk}
\includegraphics{winoRaportPls-033}

\begin{Schunk}
\begin{Sinput}
> ggplot(wine, aes(quality, volatile.acidity)) +
+   stat_summary(fun.y = median, color = "lightsteelblue4", geom = "line", size =1)
\end{Sinput}
\end{Schunk}
\includegraphics{winoRaportPls-plot2}

Polaczmy w jeden wykres:

\begin{Schunk}
\begin{Sinput}
> ggplot() + 
+   geom_line(data = by_quality, aes(x = quality, y = cit_med, color = "lightsalmon"), size = 1) + 
+   geom_line(data = by_quality, aes(x = quality, y = vol_med, color = "lightsteelblue4"), size = 1) +
+   ylab("acid") + scale_color_discrete(name = "acids:", labels = c("citric", "volatile"))
\end{Sinput}
\end{Schunk}
\includegraphics{winoRaportPls-plot3}

\subsubsection{Obserwacje}
Wraz ze wzrostem jakosci wina, do pewnego momentu wzrasta zawartosc kwasu cytrynowego 
("swiezosc"), natomiast maleje zawartosc kwasu octowego (octowy posmak)
im wyzsza jakosc, tym roznica miedzy zawartoscia kwasow maleje
zauwazmy, ze wino najwyszej jakosci (7/8) posiadaja podobna ilosc obu z tych kwasow 




\subsection{Regresja}
Zalozenia:
\begin{itemize}
  \item zaleznosc liniowa pomiedzy predyktorami a zmienna zalezna 
  \item liczba obserwacji musi byc wieksza badz rowna liczbie parametrow wyprowadzonych z analizy regresji (wspolczynniki dla predyktorow, wyraz wolny
  \item wariancja reszt powinna byc stala w calym przedziale.
  \item nie wystepuje autokorelacja reszt, rozklad reszt jest przypadkowy.
  \item reszty maja rozklad zblizony do rozkladu normalnego
  \item brak wspolliniowosci predyktorow - regresja wieloraka, wielokrotna
\end{itemize}

\subsubsection{quality - alcohol}
\begin{Schunk}
\begin{Sinput}
> ggplot(wine, aes(x=quality, y=alcohol)) + 
+   geom_point(color = c("#6633CC")) + 
+   geom_smooth(method = "lm", color = "#CC99FF")
\end{Sinput}
\end{Schunk}
\includegraphics{winoRaportPls-reg1}


\subsubsection{alcohol - density}
\begin{Schunk}
\begin{Sinput}
> ggplot(wine, aes(x=alcohol, y=density)) + 
+   geom_point(color = c("#6633CC")) + 
+   geom_smooth(method = "lm", color = "#CC99FF")
\end{Sinput}
\end{Schunk}
\includegraphics{winoRaportPls-reg2}


\subsubsection{pH - fixed.acidity}

\begin{Schunk}
\begin{Sinput}
> ggplot(wine, aes(x=pH, y=fixed.acidity)) + 
+   geom_point(color = c("#6633CC")) + 
+   geom_smooth(method = "lm", color = "#CC99FF")
\end{Sinput}
\end{Schunk}
\includegraphics{winoRaportPls-reg3}

\subsubsection{total.sulfur.dioxide - free.sulfur.dioxide}

\begin{Schunk}
\begin{Sinput}
> ggplot(wine, aes(x=total.sulfur.dioxide, y=free.sulfur.dioxide)) + 
+   geom_point(color = c("#6633CC")) + 
+   geom_smooth(method = "lm", color = "#CC99FF")
\end{Sinput}
\end{Schunk}
\includegraphics{winoRaportPls-reg4}


\subsubsection{citric.acid - fixed.acidity}
\begin{Schunk}
\begin{Sinput}
> ggplot(wine, aes(x=citric.acid, y=fixed.acidity)) + 
+   geom_point(color = c("#6633CC")) + 
+   geom_smooth(method = "lm", color = "#CC99FF")
\end{Sinput}
\end{Schunk}
\includegraphics{winoRaportPls-reg5}


\subsubsection{fixed.acidity - density}

\begin{Schunk}
\begin{Sinput}
> ggplot(wine, aes(x=fixed.acidity, y=density)) + 
+   geom_point(color = c("#6633CC")) + 
+   geom_smooth(method = "lm", color = "#CC99FF")
\end{Sinput}
\end{Schunk}
\includegraphics{winoRaportPls-reg6}


\subsection{Dopasowanie rozkladu}
\begin{Schunk}
\begin{Sinput}
> descdist(wine$alcohol, discrete = FALSE)
\end{Sinput}
\begin{Soutput}
summary statistics
------
min:  8.4   max:  14.9 
median:  10.2 
mean:  10.42298 
estimated sd:  1.065668 
estimated skewness:  0.8608288 
estimated kurtosis:  3.200029 
\end{Soutput}
\begin{Sinput}
> 
\end{Sinput}
\end{Schunk}
\includegraphics{winoRaportPls-dop1}

\begin{Schunk}
\begin{Sinput}
> descdist(wine$density, discrete = FALSE)
\end{Sinput}
\begin{Soutput}
summary statistics
------
min:  0.99007   max:  1.00369 
median:  0.99675 
mean:  0.9967467 
estimated sd:  0.001887334 
estimated skewness:  0.07128766 
estimated kurtosis:  3.934079 
\end{Soutput}
\end{Schunk}
\includegraphics{winoRaportPls-dop2}






\section{Wnioski}
Wnioski płynące z przeprowadzonej analizy, są następujące:

\begin{itemize}

\item w analizowanym zbiorze danych wystepuje wiele wartosci odstajacych 

\item wykryto wiele korelacji liniowych o dosc stosunkowo duzym wspolczynniku, m. in. 
$quality - alcohol, alcohol - density, ph - fixed.acidity, total.sulfur.dioxide - free.sulfur.dioxide$, etc

\item zawartosc kwasu cytrynowego w winie posiada 3 maksima lokalne w okolicy 0, 0.25 oraz 0.5g

\item znalezione zostaly rozklady podobne do normalnych, lecz testy nakazaly odrzucic hipoteze, jakoby byly one normalne (pH, gestosc)

\item pojawila sie ciekawa zaleznosc pomiedzy jakoscia wina, a zawartoscia kwasu octowego i kwasu cytrynowego (jednoczesnie); wino najwyzszej jakosci posiada podobna ilosc obu z tych kwasow


\end{itemize}


\end{document}
